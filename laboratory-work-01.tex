\documentclass{report} 

% Опциональные пакеты (например, для символов, математики и т.д.) 
\usepackage[utf8]{inputenc}  % Обработка UTF-8 кодировки 
\usepackage[T2A]{fontenc}    % Поддержка кириллических шрифтов 
\usepackage[russian, english]{babel}  % Поддержка русского языка 

\usepackage[a4paper,left=30mm,right=10mm,top=20mm,bottom=20mm]{geometry}
\usepackage{amsmath}         % Полезно для математических формул 
\usepackage{graphicx}        % Включение изображений 
 
\title{Реферат на тему: \\[0.5cm] \textbf{Кошки: История, породы и роль в обществе}}
\author{Студент: Аннануров Даниил Петрович \\ Группа: 1.2} 
\date{\today}                % Автоматически использует сегодняшнюю дату 
 
\begin{document} 
 
\maketitle                  % Генерация титульной страницы 
\chapter{Введение}

\section{Введение} 
Кошки (Felis catus) — это небольшие хищные млекопитающие, которые были одомашнены человеком более 9 тысяч лет назад. На протяжении всей истории, кошки играли важную роль в жизни людей, будучи как спутниками, так и помощниками в борьбе с грызунами. Сегодня кошки считаются одними из самых популярных домашних животных в мире, обладая обаянием, грацией и независимым характером.

\chapter{История одомашнивания}

\section{Происхождение}
Кошки, вероятно, были одомашнены в древнем Египте, где их почитали как священных животных. Эти животные были ценными для фермеров, так как защищали урожай от мышей и крыс. В древности кошки считались символом защиты и плодородия, и их убийство даже каралось смертью.

\section{Кошки в культурах мира}
Помимо Египта, кошки играли значительную роль и в других культурах. В Древнем Китае они считались стражами домашнего очага, а в Японии их уважали как хранителей от злых духов. В Средневековой Европе кошки часто ассоциировались с магией, что приводило к преследованию этих животных.

\chapter{Основные породы кошек}

\section{Многообразие пород}
На сегодняшний день существует более 70 официально признанных пород кошек, каждая из которых обладает уникальными чертами, будь то внешний вид, поведение или характер. Рассмотрим некоторые из самых популярных пород.

\subsection{Персидская кошка}
Персидские кошки известны своей длинной, густой шерстью и спокойным, уравновешенным характером. Они требуют особого ухода за шерстью и очень ценятся среди любителей домашних животных за свою грацию и изящество.

\subsection{Сиамская кошка}
Сиамские кошки имеют стройное тело, характерный светлый окрас с темными отметинами на мордочке, ушах, лапах и хвосте, а также ярко-голубые глаза. Они известны своим общительным и иногда даже требовательным поведением, что делает их хорошими компаньонами.

\subsection{Мейн-кун}
Мейн-кун — одна из крупнейших пород домашних кошек. Эти кошки известны своим дружелюбным и ласковым характером, а также впечатляющей внешностью с длинной шерстью и пушистым хвостом. Несмотря на свой крупный размер, они очень игривы и легко находят общий язык с детьми.

\begin{table}[h]
    \centering
    \begin{tabular}{|c|c|c|}
    \hline
    Порода & Характеристики & Происхождение \\
    \hline
    Персидская & Длинная шерсть, спокойная & Иран \\
    \hline
    Сиамская & Стройная, голубые глаза & Таиланд \\
    \hline
    Мейн-кун & Крупная, дружелюбная & США \\
    \hline
    \end{tabular}
    \caption{Примеры пород кошек}
    \label{tab:example}
\end{table}

\chapter{Роль кошек в современном обществе}

\section{Кошки как домашние питомцы}
Сегодня кошки играют важную роль как домашние питомцы, предлагая своим владельцам дружбу, эмоциональную поддержку и радость. Исследования показывают, что наличие кошки может снизить уровень стресса и тревоги у людей, а также способствовать улучшению психического здоровья.

\section{Кошки в поп-культуре}
Кошки широко представлены в литературе, кино и интернете. Такие персонажи, как Чеширский кот из «Алисы в Стране чудес» или Гарфилд, стали культовыми образами. В интернете кошки стали феноменом, с множеством мемов и видео, которые собирают миллионы просмотров.

\chapter{Заключение}

Кошки — это удивительные существа, которые принесли человечеству много пользы и радости. Они остаются важной частью культуры и быта людей. Независимо от того, предпочитаете ли вы спокойную персидскую кошку или игривого мейн-куна, каждый найдет себе питомца по душе.

\begin{figure}[h]
    \centering
    \includegraphics[width=0.5\textwidth]{image.png}
    \caption{Милый котик}
    \label{fig:example}
\end{figure}

\begin{equation} 
    E = mc^2 
\end{equation} 

\begin{align}\label{eq:myformula}
    a &= b + c \\
    d &= e + f
\end{align}

\chapter{Список литературы}
\begin{enumerate}
\item Джонсон, С. "Полное руководство по кошкам." Издательство Pet Publishing, 2020.
\item Смит, А. "Понимание вашей кошки." Издательство Feline Press, 2018.
\item Котов, И. "История одомашнивания кошек." Журнал «Домашние животные», 2019.
\item Филатов, В. "Породы кошек: краткий справочник." Издательство Зоолюб, 2017.
\end{enumerate}

\end{document}
